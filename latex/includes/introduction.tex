


Cette étude porte sur 3 bases de données medicales VAPOR, RTUPB et VPPBS. Ces trois bases fournissent un ensemble de données pré et post opératoire pour un ensemble de patients utilisant l'un des trois traitements. Sachant que les données post opératoires sont fournies sous forme d’observation sur des intervalles de temps distincts. 


\newcolumntype{M}[1]{>{\raggedright}m{#1}}
\begin{table}[!h]
\centering
\caption{Glossaire}
\begin{tabular}{|l|p{11cm}|}
\toprule
Variable   & Déscription    \tabularnewline              \\
\midrule
Age (ans)     & Age du  patient                \\ 
\hline
Comorbidité CardioVx     &  Présence de maladies  associées cardiaque ou vasculaire tel que  l'hypertension arterielle      \\
\hline
Durée traitement médical (mois)  &   N/A       \\
\hline
Porteur de sonde & le patient a une sonde urinaire avant l'intervention      \\
\hline
IPSS P.O     & International prostatic syptome score PRE OPERATOIRE = plus il est elevé plus le patient est gené          \\
\hline
QoL P.O   &  Score de qualité de vie PRE OPERATOIRE = plus il est elevé plus le patient est insatisfait                          \\
\hline
Qmax P.O (ml/s)    & Débit maximal urinaire PRE OPERATOIRE = plus il est elevé, plus la miciton est de bonne qualité 	  \\
\hline
PSA (ng/ml)    & N/A  \\
\hline
Volume prostatique (ml)    & N/A  \\
\hline
RPM     & Residu post mictionnel = quantité d'urine retrouvé dans vessie après une miction, à l'état normal elle est de 0    \\
\hline
Indication  & N/A                           \\
\hline
Anesthésie     & N/A                           \\
\hline
Evenement H.D    & Evenement hémodynamique pendant l'intervention = perturbation de la tension arterielle durant l'intervention   \\
\hline
Transfusion PerO  & Si oui ou non le patient a eu une transfusion pendant l'intervention                           \\
\hline
Temps OP     & Temps opératoire                          \\
\hline
Volume résequé (ml)   & N/A                           \\
\hline
Délai ablation (jours)l)   & Délai d'ablation de la sonde urinaire après l'intervention 	  \\
\hline
caillotage & N/A                           \\
\hline
reprise au bloc  & N/A                           \\
\bottomrule
\end{tabular}
\end{table}

	 	 	 			