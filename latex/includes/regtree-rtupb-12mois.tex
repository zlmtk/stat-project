%"###############################################
%
% Classification supervisée
%
%###############################################

% RTUPB: variables post-op à 12 mois
%
%###############################################

Dans les sections qui suivent, nous réutilisons la même méthode pour construire un arbre de régression:
pour chacune des techniques opératoires (respectivement RTUPB, VPPBS et VAPOR) et à partir des données pré-opératoires précédemment utilisées et nettoyées des variables invariantes ou incomplètes, auxquelles on ajoutera la variable à prédire, nous effectuons un tirage aléatoire pour créer deux ensembles de données:
\begin{itemize}
\item un ensemble d'apprentissage représentant 80\% des patients opérés par une technique. A partir de cet ensemble, nous inférerons un arbre de régression en utilisant le package rpart sous R.
\item un ensemble de validation représentant 20\% des patients opérés par la même technique.
\end{itemize}
Les variables IPSS et QoL seront traitées comme des variables ordinales, tandis que la variable Qmax sera
traitée comme une variable linéaire.

Pour chaque variable et chaque base, nous comparerons les résultats obtenus par un arbre de régression (inféré par élagage d'un arbre "complet") et les résultats obtenus par une forêt aléatoire d'arbres de régression.

\subsubsection{RTUPB: IPSS à 12 mois}

L'arbre de régression complet obtenu pour la variable IPSS à 12 mois (colonne IPSS\_\_4) met en {\oe}uvre 6 variables (Cf. figure~\ref{fig-rtupb-full-regtree-ipss12}).
Les feuilles de l'arbre représentent, en première ligne, la valeur prédite (la plus probable) pour IPSS à 12 mois. La ligne suivante indique les probabilités pour chacun des 3 niveaux constatés sur l'échantillon d'apprentissage pour IPSS à 12 mois, c'est-à-dire respectivement 1/2/3. Par exemple en cas de caillotage (valeur == 1), l'arbre de régression conduit à la feuille en bas à gauche et prédit une valeur IPSS à 1 avec une probabilité de 88\%.

\begin{figure}[H]
\centering
\includegraphics[width=0.75\textwidth]{../Fig/RTUPB/rtupb-full-regtree-ipss12.png}
\caption{RTUPB: Arbre de régression pour IPSS à 12 mois}
\label{fig-rtupb-full-regtree-ipss12}
\end{figure}

Afin de simplifier l'arbre, nous vérifions le facteur d'amélioration (complexity parameter) apporté par chaque embranchement de l'arbre, figure~\ref{fig-rtupb-regtree-optim-ipss12}. Nous élaguons l'arbre au niveau du coude d'inflexion de la courbe, soit environ à cp=0.13, pour obtenir un arbre simplifié  (figure~\ref{fig-rtupb-regtree-predict-ipss12}).

\begin{figure}[H]
\centering
\includegraphics[width=0.75\textwidth]{../Fig/RTUPB/rtupb-regtree-optim-ipss12.png}
\caption{RTUPB: Optimisation de l'arbre de régression pour IPSS à 12 mois}
\label{fig-rtupb-regtree-optim-ipss12}
\end{figure}

En appliquant cet arbre élagué  (figure~\ref{fig-rtupb-regtree-predict-ipss12}) sur l'ensemble de tests, nous obtenons pour chaque patient une table de probabilités pour chacune des valeurs IPSS 1/2/3/4. Les prédictions (valeurs
les plus probables) sont correctes pour tous les patients à l'exception des patients 20 et 29 pour lesquels la valeur référence IPSS à 12 mois est de 1.
Nous obtenons la table de correspondance suivante, entre valeurs de référence et valeurs de prédiction:

\begin{lstlisting}[language=R]
          prediction
validation 1 2 3 4
         1 0 2 0 0
         2 0 5 1 0
\end{lstlisting}
Soit un taux d'erreur de 37,5\%.

\begin{figure}[H]
\centering
\includegraphics[width=0.75\textwidth]{../Fig/RTUPB/rtupb-regtree-ipss12.png}
\caption{RTUPB: Arbre de régression pour IPSS à 12 mois}
\label{fig-rtupb-regtree-ipss12}
\end{figure}


\begin{figure}[H]
\centering
\includegraphics[width=0.75\textwidth]{../Fig/RTUPB/rtupb-regtree-predict-ipss12.png}
\caption{RTUPB: Prévision pour IPSS à 12 mois}
\label{fig-rtupb-regtree-predict-ipss12}
\end{figure}

En utilisant une forêt aléatoire d'arbres de régression, nous obtenons une nouvelle table de correspondance:
\begin{lstlisting}[language=R]
           prediction
validation 1 2 3 4
         1 1 1 0 0
         2 0 4 1 1
\end{lstlisting}
Soit un taux d'erreur de 37,5\%. Sous réserve qu'une marge d'erreur de prédiction d'un point est acceptable sur l'indice IPSS (hypothèse à valider par un avis médical), le taux d'erreur serait alors ramené à 12,5\%.

\subsubsection{RTUPB: QoL à 12 mois}

L'arbre de régression obtenu pour la variable QoL à 12 mois (colonne QoL\_\_4) met en {\oe}uvre cinq variables (Cf. figure~\ref{fig-rtupb-full-regtree-qol12}). Les feuilles de l'arbre représentent, en première ligne, la valeur prédite (la plus probable) pour QoL à 12 mois. La ligne suivante indique les probabilités pour chacun des 3 niveaux constatés sur l'échantillon d'apprentissage pour QoL à 12 mois, c'est-à-dire respectivement 0/1/2. 

\begin{figure}[H]
\centering
\includegraphics[width=0.75\textwidth]{../Fig/RTUPB/rtupb-full-regtree-qol12.png}
\caption{RTUPB: Arbre de régression complet pour QoL à 12 mois}
\label{fig-rtupb-full-regtree-qol12}
\end{figure}

Nous vérifions l'amélioration apportée par chaque embranchement de cet arbre de régression complet, figure~\ref{fig-rtupb-regtree-optim-qol12}. Un élagage au point d'inflexion (cp=0.15) permet d'obtenir un arbre simplifié, représenté figure~\ref{fig-rtupb-regtree-qol12}.

\begin{figure}[H]
\centering
\includegraphics[width=0.75\textwidth]{../Fig/RTUPB/rtupb-regtree-optim-qol12.png}
\caption{RTUPB: Optimisation de l'arbre de régression pour QoL à 12 mois}
\label{fig-rtupb-regtree-optim-qol12}
\end{figure}

\begin{figure}[H]
\centering
\includegraphics[width=0.75\textwidth]{../Fig/RTUPB/rtupb-regtree-qol12.png}
\caption{RTUPB: Arbre de régression élagué pour QoL à 12 mois}
\label{fig-rtupb-regtree-qol12}
\end{figure}

\begin{figure}[H]
\centering
\includegraphics[width=0.75\textwidth]{../Fig/RTUPB/rtupb-regtree-predict-qol12.png}
\caption{RTUPB: Prévision pour QoL à 12 mois}
\label{fig-rtupb-regtree-predict-qol12}
\end{figure}

\begin{figure}[H]
\centering
\includegraphics[width=0.75\textwidth]{../Fig/RTUPB/rtupb-regtree-test-qol12.png}
\caption{RTUPB: Valeurs Test pour QoL à 12 mois}
\label{fig-rtupb-regtree-test-qol12}
\end{figure}

En appliquant cet arbre sur l'ensemble de tests, nous obtenons pour chaque patient une table de probabilités (figure~\ref{fig-rtupb-regtree-predict-qol12}) pour chacune des valeurs QoL 0/1/2. On peut alors comparer les prédictions (les valeurs les plus probables) avec les valeurs de référence (figure~\ref{fig-rtupb-regtree-test-qol12}). Celles-ci sont correctes pour 4 patients sur 8, et incorrectes pour les 4 autres (20, 29, 5 et 21). Nous obtenons la table de correspondance suivante, entre valeurs de référence et valeurs de prédiction:

\begin{lstlisting}[language=R]
           prediction
validation 0 1 2
         0 0 2 1
         1 0 1 0
         2 0 0 4
\end{lstlisting}
Soit un taux d'erreur de 37,5\%.

Afin d'améliorer les résultats de prédiction, il nous faudrait répéter l'opération et utiliser une forêt d'arbres de régression. La prédiction utilisée sera alors la prédiction moyenne de l'ensemble des arbres de la forêt. Nous obtenons alors une nouvelle table de correspondance:

\begin{lstlisting}[language=R]
          prediction
validation 0 1 2
         0 1 1 1
         1 0 1 0
         2 0 0 4
\end{lstlisting}
Soit un taux d'erreur moindre~: 25\%.


\subsubsection{RTUPB: Qmax à 12 mois}

\begin{figure}[H]
\centering
\includegraphics[width=0.75\textwidth]{../Fig/RTUPB/rtupb-regtree-optim-qmax12.png}
\caption{RTUPB: Optimisation de l'élagage de l'arbre complet pour Qmax à 12 mois}
\label{fig-rtupb-regtree-optim-qmax12}
\end{figure}

L'arbre de régression obtenu pour la variable Qmax à 12 mois (colonne Qmax (ml/s)\_\_3), après élagage (Cf. figure pour le choix d'élagage optimal), met en {\oe}uvre 2 variables (Cf. figure~\ref{fig-rtupb-regtree-qmax12}). Les feuilles de l'arbre représentent, en première ligne, la valeur approximative pour Qmax à 12 mois. La ligne suivante indique le nombre de patients de l'échantillon d'apprentissage correspondant à cette feuille. 

\begin{figure}[H]
\centering
\includegraphics[width=0.75\textwidth]{../Fig/RTUPB/rtupb-regtree-qmax12.png}
\caption{RTUPB: Arbre de régression pour Qmax à 12 mois}
\label{fig-rtupb-regtree-qmax12}
\end{figure}

Pour évaluer l'arbre de régression, nous utiliserons une mesure de distance (valeur absolue de la différence) pour entre la valeur prédite et la valeur de référence pour chaque patient de l'ensemble de validation. 
La figure~\ref{fig-rtupb-regtree-test-qmax12} représente l'écart constaté entre valeurs prédites (colonne de gauche) et valeurs de référence (colonne de droite). La moyenne de ces écarts nous donne une mesure du taux d'erreur de l'arbre. En répétant l'opération pour construire une forêt d'arbres de régression, nous utiliserons alors la prédiction rendue par l'arbre avec le plus faible taux d'erreur (Cf. figure~\ref{fig-rtupb-forest-test-qmax12}).

\begin{figure}[H]
\centering
\includegraphics[width=0.75\textwidth]{../Fig/RTUPB/rtupb-regtree-test-qmax12.png}
\caption{RTUPB: Evaluation des prévisions pour Qmax à 12 mois}
\label{fig-rtupb-regtree-test-qmax12}
\end{figure}

\begin{figure}[H]
\centering
\includegraphics[width=0.75\textwidth]{../Fig/RTUPB/rtupb-forest-test-qmax12.png}
\caption{RTUPB: Evaluation des prévisions (RandomForest) pour Qmax à 12 mois}
\label{fig-rtupb-forest-test-qmax12}
\end{figure}
