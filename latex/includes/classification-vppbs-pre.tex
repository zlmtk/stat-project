%"###############################################
%
% Classification VPBBS pre 
%
%###############################################

\begin{figure}[H]
\centering
\includegraphics[width=0.90\textwidth]{../Fig/VPPBS/vppbs-elbow-pre.png}
\caption{VPPBS Maximise nb cluster / bonne classification}
\end{figure}

\begin{figure}[H]
\centering
\includegraphics[width=0.90\textwidth]{../Fig/VPPBS/vppbs-pam-k12.png}
\caption{VPPBS Nuage de points / clusters PAM k=12 }
\end{figure}

\begin{figure}[H]
\centering
\includegraphics[width=0.90\textwidth]{../Fig/VPPBS/vppbs-sil-k12-pre.png}
\caption{VPPBS silhouette/ clusters PAM k=12 }
\end{figure}

%\begin{figure}[H]
\begin{figure}[H]
\centering
\includegraphics[width=0.90\textwidth]{../Fig/VPPBS/vppbs-cah-k12-pre.png}
\caption{VPPBS CAH séparation en k=12 }
\end{figure}


%"###############################################
%
% Interpretation des trois figures VPPBS pre 
%
%#############################################






%\begin{figure}[h]
%    \begin{minipage}[c]{.46\linewidth}
%        \centering
%        \includegraphics[width=1\textwidth]{../Fig/VPPBS/vppbs-corr-matrice-pie}
%        \caption{Légende}
%    \end{minipage}
%    \hfill%
%    \begin{minipage}[c]{.46\linewidth}
%        \centering
%        \includegraphics[width=1\textwidth]{../Fig/VPPBS/vppbs-corr-matrice-pie}
%        \caption{Légende}
%    \end{minipage}
%\end{figure}


%
%##########################
%# CONCLUSION
%##########################
Dans la cas de la technique VPPBS nous pouvons observer aussi quelques clusters avec des individus similaires (observations complètements identiques) . La CAH semble indiquer qu’il existe   deux clusters distincts, et la silhouette de PAM nous propose le meilleur classement avec un nombre de 12 clusters. Ceci sera à mettre en emphase avec les profils de guérison. 

\textbf{Patients medoids: 27,2,29,4,31,32,21,22,23,24,25,26}
